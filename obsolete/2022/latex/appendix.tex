\documentclass{jsarticle}
\usepackage{fancybox,ascmac}
\usepackage{epstopdf}
\usepackage[dvipdfmx]{graphicx}
\usepackage{amsmath,amssymb}
\usepackage{bm}
\usepackage{alltt}
\usepackage{authblk}
\usepackage{fancyhdr}
\usepackage{url}
\usepackage{physics}
\usepackage[version=3]{mhchem}
\usepackage{siunitx}

\begin{document}
\section{便利なパッケージ}
\subsection{Physics}
物理系なら使うべきパッケージ.

入力が面倒な微分や行列などを簡潔に入力できるようになる.

\begin{minipage}[c]{.50\textwidth}
\begin{screen}
\small
\begin{verbatim}
\usepackage{physics}

\int y \dd{x} = \int y \dv{x}{t} \dd{t}
\pdv[2]{u}{t} = c^2 \qty(\pdv[2]{u}{x})
A = \mqty( \imat{3} )
B = \mqty(\xmat*{b}{2}{3}) 
\end{verbatim}
\end{screen}
\end{minipage}%
%\manerrarrow\hfill{}
$\Rightarrow$
\begin{minipage}{.45\textwidth}
\begin{shadebox}
\[\int y \dd{x} = \int y \dv{x}{t} \dd{t} \]
\[\pdv[2]{u}{t} = c^2 \qty(\pdv[2]{u}{x}) \]
\[A = \mqty( \imat{3} ) \]
\[B = \mqty(\xmat*{b}{2}{3})\]
\end{shadebox}
\end{minipage}
\vspace*{1mm}\\

\subsection{mhchem}
化学式の入力が簡単にできるようになる.

\begin{minipage}[c]{.50\textwidth}
\begin{screen}
\small
\begin{verbatim}
\usepackage[version=3]{mhchem}

\ce{H2O}
\ce{CrO4-}, \ce{NH4+}, \ce{Cu^2+}
\ce{Mg3Si2O5(OH)4 -> Mg3Si2O7 + 2H2O}
\ce{^{227}_{90}Th+}
\end{verbatim}
\end{screen}
\end{minipage}%
%\manerrarrow\hfill{}
$\Rightarrow$
\begin{minipage}{.45\textwidth}
\begin{shadebox}
\ce{H2O} \\
\ce{CrO4-}, \ce{NH4+},\ce{Cu^2+} \\
\ce{Mg3Si2O5(OH)4 -> Mg3Si2O7 + 2H2O}
\ce{^{227}_{90}Th+}
\end{shadebox}
\end{minipage}
\vspace*{1mm}\\
\subsection{siunitx}
単位を書くときに非常に便利.

\begin{minipage}[c]{.50\textwidth}
\begin{screen}
\small
\begin{verbatim}
\usepackage{sinuitx}

\SI{4.4}{m/s}
\SI{6.6720E11}{N m^2/ kg}
\SI{\kilo\gram\metre\per\square\second}
\end{verbatim}
\end{screen}
\end{minipage}%
%\manerrarrow\hfill{}
$\Rightarrow$
\begin{minipage}{.45\textwidth}
\begin{shadebox}
\SI{4.4}{m/s} \\
\SI{6.6720E11}{N m^2/ kg} \\
\si{\kilo\gram\metre\per\square\second}
\end{shadebox}
\end{minipage}
\end{document}