%kadai
%%%%%%%%%%%%%%%%%%%%%%%%%%%%%%%%%%%%%%%
%%%%%%%%%%%%%%%%%%%%%%%%%%%%%%%%%%%%%%%

\section{課題:数式の書き方,図の貼り込み}
\underline{/home2/takata2022/homework/homework.pdf}という文書があります。
しかしこの文書には残念ながら適当な図が貼り付けてありません。
本文を{\TeX}で作成し、
先日習ったgnuplotで適切な図を作成し
(どんな図でもいいです、また以前つくったものでも構いません)、貼り付けましょう。
ついでに、
文書の最初に課題名、名前、学籍番号、提出日も入れておきましょう。
空白の大きさ等細かい所まで配布文書をまねる必要はないですが、
図の番号付け等、最低限レポートとして人に見せられる体裁は整えましょう。

余裕がある人は図の下または次ページに何かをつけ加えてみてください(箇条書き等の、課題でまだ使っていないコマンドを使ってみる。図を横に複数枚並べてみる。など)。
\\

文書作成の際には新しいプロジェクトを立ち上げてください(この際\ref{sec:overleaf}で行った設定を忘れずに)。
出来上がったら、pdfファイルとソースファイルをダウンロードし、
slackのダイレクトメッセージで\verb+.zip+ファイルを高田に送ってください。
期限は4月20日(水)12:59(JST)とします。



\pagebreak
