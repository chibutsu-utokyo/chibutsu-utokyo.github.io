%section 2
%%%%%%%%%%%%%%%%%%%%%%%%%%%%%%%%%%%%%%%%%%%%%%%%%%%%%%%%%%%%%%%%%%%%%%%%%%%%%%%%
%%%%%%%%%%%%%%%%%%%%%%%%%%%%%%%%%%%%%%%%%%%%%%%%%%%%%%%%%%%%%%%%%%%%%%%%%%%%%%%%

\section{はじめに}
1章にしたがって使えば、とりあえず動くようにはなります。
しかし、少しは{\LaTeX}の歴史を知っておくのもよいでしょう。

\subsection{{\TeX} とは}
{\TeX}\footnote{手書きなどで{\TeX}というロゴが出力できないときは、\emph{必ず} `TeX'のようにeだけを小文字にして書きます。同様に{\LaTeX}は`LaTeX'と書きます。まあ`KinKi Kids'みたいなもんです。}というソフトは、\verb+Stanford+大学の\verb+Donald E. Knuth+教授(当時)が作成した文書組版用ソフトウェアです。
{\TeX}をどう発音するかはなかなか難しい問題なのですが、これはギ
リシャ語の$\tau$と$\epsilon$と$\chi$をくっつけった発音になるのだそうです。
日本語にも英語にも$\chi$という音はないので、だいたい「てっく」とか「てふ」
と発音します。

組版(typesetting)というのはあまり聞き慣れない言葉ですが、もともと印刷用語で、
活字を組んで紙面を構成することを指します。{\TeX}はこれをコンピュータの上で行うための
ソフトウェアです。

{\TeX}には次のような特徴があります:
\begin{itemize}
 \item {\TeX}はフリーソフトなので、無料で入手できます。インストールの方法
       も日進月歩で進化しており、ご家庭のパソコンにも簡単にインストール
       できます\footnote{TeXLiveやMacTeXなどで調べてみてください。}。
 \item WindowsやMacintosh、UNIXなど様々なプラットフォームで、まったく同
       じ出力が得られます。
 \item 特に数式の美しさは変態的です。出版の専門家でない我々が
\begin{equation}
 \frac{d}{dt} \Phi (t) = \lim_{\delta t \rightarrow 0} \frac{\Phi(t +
  \delta t) - \Phi(t)}{\delta t} \\
    =  \int_{S(t)} \left( \frac{\partial}{\partial t} \vec{B} -
		      \vec{\nabla} \times (\vec{U} \times
		      \vec{B})  \right) \cdot
    d \vec{S} = 0 \nonumber
\end{equation}
       とか書けてしまうのはすばらしいことです。
 \item `dif{}ferent'や`f{}inish'が、それぞれ`different'や`finish'となる
       ように、`ff'や`fi'は文字がくっついたデザインになります。これを
       \emph{リガチャ}(合字)といます。{\TeX}は
       これを自動で行います。
 \item \emph{カーニング}(字詰め)も自動で行い、`V{}olcano'は`Volcano'の
       ようにoがVの下に入ります。
 \item 最適な改行・最適な改ページをしてくれます。たとえば、段落の最後の1行だけが
       次のページになってしまう、というようなことは避けられます。
\end{itemize}
このようにレイアウトに関しては{\TeX}がやってくれるので、我々は文書の中身を作ることに
専念できるのです。その代わりにこのような計算にはそれなりに手間がかかりますから、
「1文字打ったら則画面に反映される」ような方法では、計算機には多大な負荷
がかかるでしょうし、我々は眼が疲れるでしょう。
そこで「ある程度できた段階でまとめて{\TeX}に処理してもらう」という方法
をとっています。

\subsection{\LaTeX とは}

裸の{\TeX}は非常に完成度の高いプログラミング言語です。完成度が高いという
ことは裏を返せば一つのことを実現するのにたいへん多くの手順が必要である
ということです。これでは実用上やや問題がありますので、基本的(primitive)な{\TeX}の
機能を組み合わせてあらかじめ機能強化をしておいたほうが実用的です。

DEC(Digital Equipment Corporation)のコンピュータ科学者
Leslie Lamportは文書作成が簡単に行え
るように{\TeX}を機能強化し、{\LaTeX}というシステムを開発しました。
{\LaTeX}の発音は{\TeX}に倣って、「らてっく」とか「らてふ」でよいと思いま
す。
{\LaTeX}は、ウェブ上で使われているHTML(HyperText Markup Languege)と同じ
ようにマークアップ言語です。したがって雰囲気は似ていて、HTMLだと
\begin{screen}
\begin{verbatim}
<CENTER>中央揃え</CENTER>
\end{verbatim}
\end{screen}
となるのが、{\LaTeX}では
\begin{screen}
\begin{verbatim}
\begin{center}中央揃え\end{center}
\end{verbatim}
\end{screen}
のようになるわけです。

{\LaTeX}の初期バージョンは2.09で、現在も書籍部で{\LaTeX}の古典的な本を見
てみれば{\LaTeX}2.09の記述を発見できると思います。その後1993年には
{\LaTeXe}というバージョンにバージョンアップしました。本書ではこのシステ
ムを説明します。

\subsection{\TeX と日本語}
(株)アスキーによって日本語化された{\TeX}の仲間は、頭にp(ピー,publishing
の意)がつきます。設定ファイルlatexmkrcにあるplatexというのは、{\LaTeX} を日本
語用に修正したものなわけです。

\pagebreak
